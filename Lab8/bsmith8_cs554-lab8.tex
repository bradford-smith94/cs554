% Bradford Smith
% bsmith8_cs554-lab8.tex
% 11/19/2017
%===============

% global document styles =======================================================
\documentclass[12pt, letterpaper]{homework}

\usepackage{color}
\usepackage{etoolbox}
\AtBeginEnvironment{quote}{\small\color{gray}}

\author{Bradford Smith}
\assignment{CS 554 Lab 8}
\title{}
\date{\today}

% document begins here =========================================================
\begin{document}
\maketitle

\section*{Scenario 1: Logging}

\begin{quote}
    In this scenario, you are tasked with creating a logging server for any number
    of other arbitrary pieces of technologies.

    Your logs should have some common fields, but support any number of
    customizeable fields for an individual log entry. You should be able to
    effectively query them based on any of these fields.

    How would you store your log entries? How would you allow users to submit log
    entries? How would you allow them to query log entries? How would you allow them
    to see their log entries? What would be your web server?
\end{quote}

Answer %TODO

\section*{Scenario 2: Expense Reports}

\begin{quote}
    In this scenario, you are tasked with making an expense reporting web
    application.

    Users should be able to submit expenses, which are always of the same data
    structure: \code{id}, \code{user}, \code{isReimbursed}, \code{reimbursedBy},
    \code{paidOn}, and \code{amount}.

    When an expense is reimbursed you will generate a PDF and email it to the user
    who submitted the expense.

    How would you store your expenses? What web server would you choose, and why?
    How would you handle the emails? How would you handle PDF generation? How are
    you going to handle all the templating for the web application?
\end{quote}

Answer %TODO

\section*{Scenario 3: A Twitter Streaming Safety Service}

\begin{quote}
    In this scenario, you are tasked with creating a service for you local Police
    Department that keeps track of Tweets within your area and scans for keywords to
    trigger an investigation.

    This application comes with several parts:
    \begin{itemize}
        \item An online website to CRUD combinations of keywords to add to your
            trigger. For example, it would alert when a Tweet contains the words
            (\code{fight} or \code{drugs}) AND (\code{SmallTown USA HS} or
            \code{SMUHS}).
        \item An email alerting system to alert different officers depending on the
            contents of the Tweet, who Tweeted it, etc.
        \item A text alert system to inform officers for critical triggers (triggers
            that meet a combination that is marked as extremely important to note).
        \item A historical database to view possible incidents (Tweets that
            triggered an alert) and to mark its investigation status.
        \item A historical log of \textit{all} Tweets to retroactively search
            through.
        \item A streaming, online incident report. This would allow you to see
            Tweets as they are parsed and see their threat level. This updates in
            real time.
        \item A long term storage of all the media used by any Tweets in your area
            (pictures, snapshots of the URL, etc).
    \end{itemize}

    Which Twitter API do you use? How would you build this so its expandable to
    beyond your local precinct? What would you do to make sure that this system is
    constantly stable? What would be you web server technology? What databases would
    you use for triggers? For the historical log of Tweets? How would you handle the
    real time, streaming incident report? How would you handle storing all the media
    that you have to store as well? What web server technology would you use?
\end{quote}

Answer %TODO

\section*{Scenario 4: A Mildly Interesting Mobile Application}

\begin{quote}
    In this scenario, you are tasked with creating the web server side for a mobile
    application where people take pictures of mildly interesting things and upload
    them. The mobile application allows users to see mildly interesting pictures in
    their geographical location.

    Users must have an account to use this service. Your backend will effectively
    amount to an API and a storage solution for CRUD users, CRUD `interesting
    events', as well as an administrative dashboard for managing content.

    How would you handle the geospatial nature of your data? How would you store the
    images, both for the long term, cheap storage and for short term, fast
    retrieval? What would you write your API in? What would be your database?
\end{quote}

Answer %TODO

\end{document}

